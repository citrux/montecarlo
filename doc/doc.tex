\documentclass[article]{ncc}
\usepackage[T2A]{fontenc}
\usepackage[russian]{babel}
\author{citrux}
\title{Монте-Карло каждый день}
\usepackage{paratype}
\renewcommand*\familydefault{\sfdefault} %% Only if the base font of the document is to be sans serif
\usepackage{setspace}
\renewcommand{\vec}[1]{\boldsymbol{\mathrm{#1}}}
\begin{document}
\maketitle
\onehalfspacing

\section{Биндер и Хеерман <<Моделирование методом Монте-Карло в статистической физике>>}
\subsection{День 1. Простая выборка}

В статфизике для термодинамического среднего наблюдаемой величины в случае канонического ансамбля можем записать
\begin{equation}
    \langle A(\vec(x)) \rangle_T = \frac{1}{Z}\int d\vec{x} \exp[-\mathcal{H}(\vec{x})/k_BT]A(\vec{x}),
    \quad Z=d\vec{x} \exp[-\mathcal{H}(\vec{x})/k_BT]
\end{equation}

Применение метода Монте-Карло в равновесной статистической механике основано на идее аппроксимации точного выражения, где интегрирование ведется по всему фазовому пространству, используя суммирование по некоторому множеству точек фазового пространства:
\begin{equation}
\overline{A(\vec{x})}=\frac{\sum\limits_{l=1}^{M} \exp[-\mathcal{H}(\vec{x}_l)/k_BT]A(\vec{x}_l)}
                              {\sum\limits_{l=1}^{M} \exp[-\mathcal{H}(\vec{x}_l)/k_BT]}
\end{equation}
В пределе \(M \to \infty\) сумма должна аппроксимировать интеграл.
Только в отличие от стандартных процедур вычисления интегралов, когда точки выбираются на регулярной сетке, здесь предпочтительнее выбирать точки случайно.

Дело в том, что в многомерной сетке большинство точек находятся на поверхности гиперкуба, а внутри точек почти нет. Для получения более однородного распределения точек можно использовать псевдослучайные числа. Описанный здесь метод М-К основан на идее простой выборки.

Пример: простое случайное блуждание и блуждание без самопересечений

Преимущества простой выборки:
\begin{enumerate}
    \item за один цикл моделирования мы проучаем информацию по всему диапазону значений длины цепи и широкому диапазону температур
    \item отдельные конфигурации статистически независимы, и для их обработки применимы обычные методы анализа погрешностей
\end{enumerate}

Недостатки:
\begin{enumerate}
    \item Генерируются состояния с малым статвесом, что приводит к падению эффективности
\end{enumerate}

\subsection{День2. Выборка по значимости}

Если состояния \(\vec{x}_l\) выбраны в соответствии с некоторой вероятностью \(P(\vec{x}_l)\), то термодинамическое усреднение выполняется следующим образом:
\begin{equation}
\overline{A(\vec{x})}=\frac{\sum\limits_{l=1}^{M} \exp[-\mathcal{H}(\vec{x}_l)/k_BT]A(\vec{x}_l)/P(\vec{x}_l)}
                              {\sum\limits_{l=1}^{M} \exp[-\mathcal{H}(\vec{x}_l)/k_BT]/P(\vec{x}_l)}
\end{equation}
\end{document}