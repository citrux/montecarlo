\documentclass[article]{ncc}
\usepackage[T2A]{fontenc}
\usepackage[russian]{babel}
\author{citrux}
\title{Монте-Карло каждый день}
\usepackage{setspace}
\renewcommand{\vec}[1]{\boldsymbol{#1}}
\begin{document}
\maketitle
\tableofcontents
\onehalfspacing

\section{Биндер и Хеерман <<Моделирование методом Монте-Карло в статистической физике>>}
\subsection{День 1. Простая выборка}

В статфизике для термодинамического среднего наблюдаемой величины в случае канонического ансамбля можем записать
\begin{equation}
    \langle A(\vec(x)) \rangle_T = \frac{1}{Z}\int d\vec{x} \exp[-\mathcal{H}(\vec{x})/k_BT]A(\vec{x}),
    \quad Z=d\vec{x} \exp[-\mathcal{H}(\vec{x})/k_BT]
\end{equation}

Применение метода Монте-Карло в равновесной статистической механике основано на идее аппроксимации точного выражения, где интегрирование ведется по всему фазовому пространству, используя суммирование по некоторому множеству точек фазового пространства:
\begin{equation}
\overline{A(\vec{x})}=\frac{\sum\limits_{l=1}^{M} \exp[-\mathcal{H}(\vec{x}_l)/k_BT]A(\vec{x}_l)}
                              {\sum\limits_{l=1}^{M} \exp[-\mathcal{H}(\vec{x}_l)/k_BT]}
\end{equation}
В пределе \(M \to \infty\) сумма должна аппроксимировать интеграл.
Только в отличие от стандартных процедур вычисления интегралов, когда точки выбираются на регулярной сетке, здесь предпочтительнее выбирать точки случайно.

Дело в том, что в многомерной сетке большинство точек находятся на поверхности гиперкуба, а внутри точек почти нет. Для получения более однородного распределения точек можно использовать псевдослучайные числа. Описанный здесь метод М-К основан на идее простой выборки.

Пример: простое случайное блуждание и блуждание без самопересечений

Преимущества простой выборки:
\begin{enumerate}
    \item за один цикл моделирования мы проучаем информацию по всему диапазону значений длины цепи и широкому диапазону температур
    \item отдельные конфигурации статистически независимы, и для их обработки применимы обычные методы анализа погрешностей
\end{enumerate}

Недостатки:
\begin{enumerate}
    \item Генерируются состояния с малым статвесом, что приводит к падению эффективности
\end{enumerate}

\subsection{День 2. Выборка по значимости}

Если состояния \(\vec{x}_l\) выбраны в соответствии с некоторой вероятностью \(P(\vec{x}_l)\), то термодинамическое усреднение выполняется следующим образом:
\begin{equation}
\overline{A(\vec{x})}=\frac{\sum\limits_{l=1}^{M} \exp[-\mathcal{H}(\vec{x}_l)/k_BT]A(\vec{x}_l)/P(\vec{x}_l)}
                              {\sum\limits_{l=1}^{M} \exp[-\mathcal{H}(\vec{x}_l)/k_BT]/P(\vec{x}_l)}
\end{equation}
Наиболее естественно выбрать \( P(\vec{x}_l) \propto \exp[-\mathcal{H}(\vec{x}_l)/k_BT] \), тогда поиск среднего значения сводится к простому арифметическому усреднению:
\begin{equation}
    \overline{A(\vec{x})} = \frac{1}{M}\sum_{l=1}^M A(\vec{x}_l)
\end{equation}
Метрополис и др. развили идею, которая состоит не в последовательном выборе состояний с заданным распределением независимо одного от другого, а в построении марковского процесса, когда каждое состояние конструируется из предыдущего с помощью соответствующей вероятности перехода \( W(\vec{x}_l \to \vec{x}_{l+1}) \). Они показали, что можно выбрать вероятность перехода такой, чтобы при \(M\to\infty\)
\begin{equation}
    P(\vec{x}_l) \to P_{eq}(\vec{x}_l) = \frac{1}{Z}\exp\left(-\frac{\mathcal{H}(\vec{x}_l)}{k_BT}\right)
\end{equation}
Для этого достаточно выполнения принципа детального равновесия
\begin{equation}
    P_{eq}(\vec{x}_l)W(\vec{x}_l \to \vec{x}_{l'}) = P_{eq}(\vec{x}_{l'})W(\vec{x}_{l'} \to \vec{x}_l),
\end{equation}
\begin{equation}
    \frac{W(\vec{x}_l \to \vec{x}_{l'})}{W(\vec{x}_{l'} \to \vec{x}_l)} = \exp\left(-\frac{\delta\mathcal{H}}{k_BT}\right),\quad \delta\mathcal{H} = \mathcal{H}(\vec{x}_{l'}) - \mathcal{H}(\vec{x}_l)
    \label{eq:equilibrium_ratio}
\end{equation}
Уравнение \ref{eq:equilibrium_ratio} оставляет некоторый произвол в выборе конкретного вида \( W \). Наиболее распространены следующие 2 варианта:
\begin{equation}
    W(\vec{x}_l \to \vec{x}_{l'}) = \frac{\tau_s^{-1}}{2}\left(1-\th\frac{\delta\mathcal{H}}{2k_BT}\right),
\end{equation}
\begin{equation}
    W(\vec{x}_l \to \vec{x}_{l'}) = \left\{
        \begin{array}{ll}
            \tau_s^{-1}\exp(-\delta\mathcal{H}/k_BT),& \delta\mathcal{H}>0,\\
            \tau_s^{-1}, & \text{иначе}.
        \end{array}
    \right.
\end{equation}

\subsection{День 3. Программная реализация метода Монте-Карло. Динамическая интерпретация выборки}
\end{document}